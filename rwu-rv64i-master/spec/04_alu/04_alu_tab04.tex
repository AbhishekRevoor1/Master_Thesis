Table \ref{tab:asAlu04} shows the internal signals of the ALU. It are top level signals only. This signals are required to connect the sub-blocks. The table shows the signal name, the width, the driver, the target and a short explanation.
\begin{table}[H]
\caption{Internal Signals}
\label{tab:asAlu04}
%\begin{table}[ht]
\centering
\begin{tabularx}{\textwidth}{|l |l |X |X |X|}
  \hline
  Signal & Width & Source & Target & Explanation \\
  \hline
  \hline
  sum\_s & 64 & data01\_e, cinvb\_s, aluSel\_i[0] & aluResult\_o & The sum of the two data inputs data01\_e and data02\_e. The second data input could be negative and will 2-complemented (cinvb\_s is the inversion when aluSel\_i[0] is 1, aluSel\_i[0] also acts as the plus 1).  \\
  \hline
  start\_tx\_s & 1 & logic & uart\_top\_br\_e & The transmission of a byte via the UART is started. Pulsed logic. \\
  \hline
  asciit\_s & 8 & sine\_s & uart\_top\_br\_e & The information, which should be send via the UART. \\
  \hline
\end{tabularx}
\end{table}
