Table \ref{tab:asAlu01} shows the input and outputs of the ALU0. This are all primary inputs and outputs. There is no bi-directional, tristate or open drain pin. All pins are high-active except the pins with a \_n\_ in its name.
\begin{table}[H]
\caption{ALU0 I/O}
\label{tab:asAlu01}
%\begin{table}[ht]
\centering
\begin{tabularx}{\textwidth}{|l |l |l |X|}
  \hline
  Pin & Dir. & Wd. & Explanation \\
  \hline
  \hline
  data01\_i & in & 64 & Source data input. Comes either from the register file or the program conter (PC). Swiched by signal ``aluSrcA''. Source one for all ALU operators. \\
  \hline
  data02\_i & in & 64 & Source data input. Comes either from the register file or the immediate generator. Swiched by signal ``aluSrcB''. Source two for all ALU operators. \\
  \hline
  aluSel\_i & in & 5 & Selects the ALU operator. \\
  \hline
  aluZero\_o & out & 1 & Zero flag of the ALU. Not buffered. \\
  \hline
  aluNega\_o & out & 1 & Negative flag of the ALU. Not buffered. \\
  \hline
  aluCarr\_o & out & 1 & Carry flag of the ALU. Not buffered. \\
  \hline
  aluOver\_o & out & 1 & Overflow flag of the ALU. Not buffered. \\
  \hline
  aluResult\_o & out & 64 & The result of each ALU operation.  \\
  \hline
\end{tabularx}
\end{table}
