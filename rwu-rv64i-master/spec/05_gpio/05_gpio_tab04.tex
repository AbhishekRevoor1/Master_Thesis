
Table \ref{tab:asGpioPer04a} shows the input and outputs of the GPIO kernel.
\begin{table}[H]
\caption{GPIO I/O}
\label{tab:asGpioPer04a}
\centering
\begin{tabularx}{\textwidth}{|l |l |l |X|}
  \hline
  Pin & Direction & Width & Explanation \\
  \hline
  \hline
  clk\_i & in & 1 & System clock \\
  \hline
  rst\_i & in & 1 & System reset. Active high. \\
  \hline
  addr\_i(3:0) & in & 4 & From GPIO-BPI. Address for external GPIO devices. \\
  \hline
  data\_i(7:0) & in & 8 &  From GPIO-BPI. GPIO outputs for one block. \\
  \hline
  en\_i & in & 1 &  From GPIO-BPI. Enables a write to a GPIO block. \\
  \hline
  gpio\_o & out & 8 &  To output pins. GPIO outputs for one block, delayed by one clock and enabled by en\_i. \\
  \hline
  gpioAdr\_o & out & 4 &  To output pins. Address for external GPIO devices, delayed by one clock and enabled by en\_i.   \\
  \hline
  cs\_o & out & 1 &  To output pins. Chip select for outside GPIO blocks.   \\
  \hline
\end{tabularx}
\end{table}
