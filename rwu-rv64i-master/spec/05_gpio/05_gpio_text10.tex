In the following the registers of a GPIO peripheral are listed and described. Some registers’ layout or availability depends on the generic feature set implemented for that GPIO’s instantiation. Such registers and concerned bit fields are denoted by the use of italic style.

\subsubsection{GPIO Register Overview}

\begin{table}[H]
\caption{GPIO Register List}
\label{tab:asGpioPer04}
\centering
\begin{tabularx}{\textwidth}{|l |X |l|}
  \hline
  Register Group & Register Name & Register Symbol \\
  \hline
  \hline
  System Registers & Clock Control Register & \textbf{CLC} \\
  & Identification Register & \textbf{ID} \\
  \hline
  Special  & Run Control Register & \textbf{RUN\_CTRL} \\
  Function & Modemstatus Set Register & \textbf{MSS\_SET} \\
  Registers & Modemstatus Clear Register & \textbf{MSS\_CLR} \\
  \hline
\end{tabularx}
\end{table}


\subsubsection{Read-Write Features}
\begin{itemize}
  \item w: writable bit, by SW
  \item r: readable bit, by SW
  \item h: bit will be set by HW only
\end{itemize}

\paragraph{GPIO\_ID} -:
Identification register of this peripheral. A write will have no effect, a read will return the fixed ID. The ID will be defined on chip level.

\asregkopf{Clock control register (in BPI)}{$00_H$}
\asregeight{ '-' & clk\_ sel & '-' & '-' & kernel\_ clk\_ disable & '-' & '-' & bus\_ clk\_ disable}
           {'-' & rw & '-' & '-' & rw & '-' & '-' & rw}
           
\asregkopf{ID register (in BPI)}{$DEADBEEFDEADBEEF_H$}
\asregfourxsixteen{id 63&&&&&&&&&&&&&&&}
                             {r&r&r&r&r&r&r&r&r&r&r&r&r&r&r&r}
                             {id 47&&&&&&&&&&&&&&&}
                             {r&r&r&r&r&r&r&r&r&r&r&r&r&r&r&r}
                             {id 31&&&&&&&&&&&&&&&}
                             {r&r&r&r&r&r&r&r&r&r&r&r&r&r&r&r}
                             {id 15&&&&&&&&&&&&&&&}
                             {r&r&r&r&r&r&r&r&r&r&r&r&r&r&r&r}

\asregdesc{reserved & 7:5,2:1 & rw & not implemented, a write has no effect, a read returns a '0' \\
           \hline
           clk\_sel & 6 & rw & Selects the clock source for the UART kernel:
           \begin{itemize}
             \item '1': kernel\_clk\_i (kernel clock) drives the functional part
             \item '0': bus\_clk\_i (bus clock) drives the functional part
           \end{itemize}\\
           \hline
           kernel\_clk\_disable & 3 & rw & disables the kernel clock for the block.\\
           \hline
           bus\_clk\_disable & 0 & rw & disables the bus clock for the block.}
