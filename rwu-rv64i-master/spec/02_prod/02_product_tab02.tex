Table \ref{tab:asTopt02} shows the internal signals of the chip. It are top level signals only. This signals are required to connect the sub-blocks. The table shows the signal name, the width, the driver, the target and a short explanation.
\begin{table}[H]
\caption{Internal Signals}
\label{tab:asTopt02}
%\begin{table}[ht]
\centering
\begin{tabularx}{\textwidth}{|l |l |l |X |X|}
  \hline
  Signal & Width & Source & Target & Explanation \\
  \hline
  \hline
  txrdy\_s & 1 & uart\_top\_br\_e & - & The transmission of a byte via the UART is done. Pulsed logic. \\
  \hline
  start\_tx\_s & 1 & logic & uart\_top\_br\_e & The transmission of a byte via the UART is started. Pulsed logic. \\
  \hline
  asciit\_s & 8 & sine\_s & uart\_top\_br\_e & The information, which should be send via the UART. \\

  \hline
  cnt\_dn\_s & 1 & fsm\_ls & cnt & Down count indicator for the counter. Pulsed logic.\\
  \hline
  clr\_s & 1 & control\_fsm & fsm\_ls, cnt, reg & The synchronized input signal clr\_i. It clears the three blocks. Pulsed logic.\\
  \hline
  cnt\_s & 6 & cnt & reg, control\_fsm & The number of persons in the room. \\
  \hline
  cntr\_s & 6 & reg & uart\_tx, s3\_if & The stored version of cnt\_s. \\
  \hline
  br\_s & 1 & br & uart\_tx & It indicates the baud rate time interval. Pulsed logic. \\
  \hline
  start\_s & 1 & control\_fsm & uart\_tx & It starts the UART transmission. Pulsed logic. \\
  \hline
  finish\_s & 1 & uart\_tx & control\_fsm & It indicates, that the UART transmission is done. Active high. \\
  \hline
  action\_s & 2 & cnt & control\_fsm & It indicates an action in the door. action\_s{0}: person entered, action\_s(1): person left. Pulsed logic. \\
  \hline
  en\_s & 1 & control\_fsm & reg & Store a new value. Pulsed logic. \\
  \hline
  sndS3\_s & 2 & control\_fsm & s3\_if & Send out the counter value and the action. sndS3\_s{0}: person entered, sndS3\_s(1): person left. Pulsed logic. \\
  \hline
  finishS3\_s & 1 & s3\_if & control\_fsm & The information to S3 has been send. Active high.\\
  \hline
\end{tabularx}
\end{table}
