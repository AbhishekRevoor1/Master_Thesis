In a digital environment it is required to insert the time to the oscillation. One have to implement a Sample\&Hold element, here it is called TimerSH. This element delivers the time between the sine samples. This time ist needed by a block (nrSamples) which counts the eight stored samples and forwards this value to the next block (quadrant) which counts the four quadrants of the sine oscillation. This three values will be needed by the sine generator (\textbf{SINE-GEN}).

In the register block (\textbf{Registers}) the registers will be loaded and stored.

On the PC side, get the information and display it. The programming language is C++.
